%%%%% For comments:
\newcommand{\ignorethis } [1] {}
\newcommand{\smallnote  } [1] {{\small \{#1\}}}
\newcommand{\bignote    } [1] {\begin{quote} \textbf{Note:\ }
                               \slshape #1 \end{quote}}
\newcommand{\todo}[1]{\ignorethis{\textcolor{red}{[{\bf TODO}: #1]}}}
\newcommand{\redund}[1]{#1}

%%%%% For referencing things:
\newcommand{\chapnum    } [1] {\ref{#1}}
\newcommand{\appnum     } [1] {\ref{#1}}
\newcommand{\sectnum    } [1] {\ref{#1}}
\newcommand{\tblnum     } [1] {\ref{#1}}
\newcommand{\fignum     } [1] {\ref{#1}}
\newcommand{\eqnnum     } [1] {\mbox{(\ref{#1})}}
\newcommand{\chap       } [1] {Chapter~\chapnum{#1}}
\newcommand{\chaps      } [1] {Chapters~\chapnum{#1}}
\newcommand{\app        } [1] {Appendix~\appnum{#1}}
\newcommand{\apps       } [1] {Appendices~\appnum{#1}}
\newcommand{\sect       } [1] {Section~\sectnum{#1}}
\newcommand{\sects      } [1] {Sections~\sectnum{#1}}
\newcommand{\tbl        } [1] {Table~\tblnum{#1}}
\newcommand{\tbls       } [1] {Tables~\tblnum{#1}}
\newcommand{\fig        } [1] {Figure~\fignum{#1}}
\newcommand{\figs       } [1] {Figures~\fignum{#1}}
\newcommand{\eqn        } [1] {equation~\eqnnum{#1}}
\newcommand{\eqns       } [1] {equations~\eqnnum{#1}}

%%%%% Latin and language:
%% \newcommand{\etal       }     {\textit{et~al.}} old; not like ACM style
\newcommand{\etal       }     {{et~al.}}
\newcommand{\apriori    }     {\textit{a~priori}}
\newcommand{\aposteriori}     {\textit{a~posteriori}}
\newcommand{\perse      }     {\textit{per~se}}
\newcommand{\cf         }     {\textit{cf.}}
\newcommand{\eg         }     {{e.g.}}
\newcommand{\Eg         }     {{E.g.}}
\newcommand{\ie         }     {{i.e.}}
\newcommand{\Ie         }     {{I.e.}}
\newcommand{\vs         }     {{vs.}}
\newcommand{\naive      }     {{na\"{\i}ve}}

%%%%% Math symbols:
\newcommand{\Identity   }     {\mat{I}}
\newcommand{\Zero       }     {\mathbf{0}}
\newcommand{\Reals      }     {{\textrm{I\kern-0.18em R}}}
\newcommand{\isdefined  }     {\mbox{\hspace{0.5ex}:=\hspace{0.5ex}}}
%\newcommand{\implies    }     {\Longrightarrow}
\newcommand{\texthalf   }     {\ensuremath{\textstyle\frac{1}{2}}}
\newcommand{\half       }     {\ensuremath{\frac{1}{2}}}
\newcommand{\third      }     {\ensuremath{\frac{1}{3}}}
\newcommand{\fourth      }    {\ensuremath{\frac{1}{4}}}

%%%%% Math modifiers:
\renewcommand{\vec      } [1] {{\text{\boldmath $\mathbit{#1}$}}}
\newcommand{\mat        } [1] {{\text{\boldmath $\mathbit{#1}$}}}
\newcommand{\Approx     } [1] {\widetilde{#1}}
\newcommand{\change     } [1] {\mbox{{\footnotesize $\Delta$} \kern-3pt}#1}

%%%%% Math functions:
\newcommand{\Order      } [1] {O(#1)}
\newcommand{\set        } [1] {{\lbrace #1 \rbrace}}
\newcommand{\floor      } [1] {{\lfloor #1 \rfloor}}
\newcommand{\ceil       } [1] {{\lceil  #1 \rceil }}
\newcommand{\inverse    } [1] {{#1}^{-1}}
\newcommand{\transpose  } [1] {{#1}^\mathrm{T}}
\newcommand{\invtransp  } [1] {{#1}^{-\mathrm{T}}}

%%%%% Math functions with small (fixed) and large (expandable) forms:
\newcommand{\abs        } [1] {{| #1 |}}
\newcommand{\Abs        } [1] {{\left| #1 \right|}}
\newcommand{\norm       } [1] {{\| #1 \|}}
\newcommand{\Norm       } [1] {{\left\| #1 \right\|}}
\newcommand{\pnorm      } [2] {\norm{#1}_{#2}}
\newcommand{\Pnorm      } [2] {\Norm{#1}_{#2}}
\newcommand{\inner      } [2] {{\langle {#1} \, | \, {#2} \rangle}}
\newcommand{\Inner      } [2] {{\left\langle \begin{array}{@{}c|c@{}}
                               \displaystyle {#1} & \displaystyle {#2}
                               \end{array} \right\rangle}}


%%%%% Paper-specific stuff:

% reduce hyphenation (slay the hyper hyphenator with 2000)
\pretolerance 800

% These variables are for width and height and gaps in figures:
% set with something like: \setlength{\h}{1cm}
\newlength{\w}
\newlength{\h}
\newlength{\x}


\definecolor{darkred}{rgb}{0.7,0.1,0.1}
\definecolor{darkgreen}{rgb}{0.1,0.7,0.1}
\definecolor{cyan}{rgb}{0.7,0.0,0.7}
\definecolor{dblue}{rgb}{0.2,0.2,0.8}
\newcommand{\Mark}[1]{\textcolor{red}{{\slshape #1}}}
\newcommand{\Adam}[1]{\textcolor{darkgreen}{\textbf{Adam:} {\slshape #1}}}
\newcommand{\Steve}[1]{\textcolor{cyan}{\textbf{Steve:} {\slshape #1}}}
\newcommand{\note}[1]{\textcolor{darkgreen}{\textbf{Note:} {\slshape #1}}}

\newcommand{\argmin}{\operatornamewithlimits{argmin}}

\newcommand{\mytilde}{\raise.17ex\hbox{$\scriptstyle\sim$}}
